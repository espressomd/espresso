% Copyright (C) 2010,2012,2013,2014,2015,2016 The ESPResSo project
% Copyright (C) 2002,2003,2004,2005,2006,2007,2008,2009,2010 
%   Max-Planck-Institute for Polymer Research, Theory Group
%  
% This file is part of ESPResSo.
%   
% ESPResSo is free software: you can redistribute it and/or modify it
% under the terms of the GNU General Public License as published by the
% Free Software Foundation, either version 3 of the License, or (at your
% option) any later version.
%  
% ESPResSo is distributed in the hope that it will be useful, but
% WITHOUT ANY WARRANTY; without even the implied warranty of
% MERCHANTABILITY or FITNESS FOR A PARTICULAR PURPOSE.  See the GNU
% General Public License for more details.
%  
% You should have received a copy of the GNU General Public License
% along with this program.  If not, see <http://www.gnu.org/licenses/>.
%
\chapter{Auxilliary commands}
\label{chap:aux}

\todo{Missing commands: 
  Probably all from \keyword{scripts/auxiliary.tcl}?}

\section{Finding particles and bonds}

\subsection{\texttt{countBonds}}
\begin{essyntax}
  countBonds \var{particle_list}
\end{essyntax}
Returns a Tcl-list of the complete topology described by
\var{particle\_list}, which must have the same format as the output of
the command \lit{part} (see section \vref{tcl:part}).

The output list contains only the particle id and the corresponding
bonding information, thus it looks like \eg{}
\begin{tclcode}
{106 {0 107}} {107 {0 106} {0 108}} {108 {0 107} {0 109}} ...
{210 {0 209} {0 211}} {211 {0 210}} 212 213 ... 
\end{tclcode}
for a single chain of 106 monomers between particle 106 and 211, with
additional loose particles 212, 213, ... (\eg{} counter-ions).  Note,
that the \lit{part} command stores any bonds only with the particle of
lower particle number, which is why \codebox{[part 109]} would only
return \codebox{{... bonds {{0 110}}}}, therefore not revealing the
bond between particle 109 and (the preceding) particle 108, while
\lit{countBonds} would return all bonds particle 109 participates in.

\subsection{\texttt{findPropPos}}
\begin{essyntax}
  findPropPos \var{particle_property_list} \var{property}
\end{essyntax}
Returns the index of \var{property} within
\var{particle_property_list}, which is expected to have the same
format as \codebox{[part \var{particle_id}]}. If \var{property} is not
found, \lit{-1} is returned.

This function is useful to access certain properties of particles
without hard-wiring their index-position, which might change in future
releases of \lit{part}.

\minisec{Example}
\begin{tclcode}
[lindex [part $i] [findPropPos [part $i] type]]
\end{tclcode}
This returns the particle type id of particle \var{i} without fixing
where exactly that information has to be in the output of 
\codebox{[part \$i]}.

\subsection{\texttt{findBondPos}}
\begin{essyntax}
  findBondPos \var{particle_property_list}
\end{essyntax}

Returns the index of the bonds within \var{particle_property_list},
which is expected to have the same format as \codebox{[part
  \var{particle\_number}]}; hence its output is the same as
\codebox{[findPropPos \var{particle_property_list} bonds]}. If the
particle does not have any bonds, \lit{-1} is returned.

\subsection{\texttt{timeStamp}}
\begin{essyntax}
  timeStamp \var{path} \var{prefix} \var{postfix} \var{suffix}
\end{essyntax}
Modifies the filename contained within \var{path} to be preceded by a
\var{prefix} and having \var{postfix} before the \var{suffix}; \eg
\begin{tclcode}
  timeStamp ./scripts/config.gz DH863 001 gz
\end{tclcode}
returns \lit{./scripts/DH863_config001.gz}.  If \var{postfix} is $-1$,
the current date is used in the format \lit{\%y\%m\%d}. This would
results in \lit{./scripts/DH863\_config021022.gz} on October 22nd,
2002.

\section{Additional Tcl math-functions}
The following procedures are found in scripts/ABHmath.tcl.
\begin{itemize}
 \item
  CONSTANTS
  \begin{itemize}
   \item
\begin{code}
PI
\end{code}
    returns $\pi$ with 16 digits precision.
   \item
\begin{code}
KBOLTZ
\end{code}
Returns Boltzmann constant in Joule/Kelvin
   \item
\begin{code}
ECHARGE
\end{code}
    Returns elementary charge in Coulomb
   \item
\begin{code}
NAVOGADRO
\end{code}
    Returns Avogadro number
   \item
\begin{code}
SPEEDOFLIGHT
\end{code}
    Returns speed of light in meter/second
   \item
\begin{code}
EPSILON0
\end{code}
Returns dielectric constant of vaccum in Coulomb\^2/(Joule meter)
   \item
\begin{code}
ATOMICMASS
\end{code}
    Returns the atomic mass unit u in kilogramms
  \end{itemize}
 \item
  MATHEMATICAL FUNCTIONS
  \begin{itemize}
   \item
\begin{code}
sqr <arg>
\end{code}
    returns the square of \var{arg}.
   \item
\begin{code}
min <arg1> <arg2>
\end{code}
    returns the minimum of \var{arg1} and \var{arg2}.
   \item
\begin{code}
max <arg1> <arg2>
\end{code}
    returns the maximum of \var{arg1} and \var{arg2}.
   \item
\begin{code}
sign <arg>
\end{code}
returns the signum-function of \var{arg}, namely +1 for \var{arg}
$>0$, -1 for $<0$, and =0 otherwise.
  \end{itemize}
 \item
  RANDOM FUNCTIONS
  \begin{itemize}
   \item
\begin{code}
gauss\_random
\end{code}
returns random numbers which have a Gaussian distribution
   \item
\begin{code}
dist\_random <dist> [max]
\end{code}
returns random numbers in the interval $[0,1]$ which have a
distribution according to the distribution function p(x) \var{dist}
which has to be given as a tcl list containing equally spaced values
of p(x). If p(x) contains values larger than 1 (default value of max)
the maximum or any number larger than that has to be given \var{max}.
This routine basically takes the function p(x) and places it into a
rectangular area ([0,1],[0,max]). Then it uses to random numbers to
specify a point in this area and checks wether it resides in the area
under p(x). Attention: Since this is written in tcl it is probably not
the fastest way to do this!
   \item
\begin{code}
vec\_random [len]
\end{code}
returns a random vector of length \var{len} (uniform distribution on a
sphere) This is done by chosing 3 uniformly distributed random numbers
$[-1,1]$ If the length of the resulting vector is $<= 1.0$ the vector
is taken and normalized to the desired length, otherwise the procedure
is repeated until succes. On average the procedure needs 5.739 random
numbers per vector. (This is probably not the most efficient way, but
it works!) Ask your favorit mathematician for a proof!
   \item
\begin{code}
phivec\_random <v> <phi> [len]
\end{code}
    return a random vector at angle \var{phi} with \var{v} and length \var{len}
  \end{itemize}
 \item
  PARTICLE OPERATIONS

  Operations involving particle positions. The parameters \var{pi} can
  either denote the particle identity (then the particle position is
  extracted with the The part command command) or the particle
  position directly When the optional \var{box} parameter for minimum
  image conventions is omited the functions use the the \codebox{setmd
    box\_l} command.
  \begin{itemize}
   \item
\begin{code}
bond\_vec <p1> <p2>
\end{code}
    Calculate bond vector pointing from particles \var{p2} to \var{p1}
    return = (\var{p1}.pos - \var{p2}.pos)
   \item
\begin{code}
bond\_vec\_min <p1> <p2> [box]
\end{code}
    Calculate bond vector pointing from particles \var{p2} to \var{p1}
    return = MinimumImage(\var{p1}.pos - \var{p2}.pos)
   \item
\begin{code}
bond\_length <p1> <p2>
\end{code}
    Calculate bond length between particles \var{p1} and \var{p2}
   \item
\begin{code}
bond\_length\_min <p1> <p2> [box]
\end{code}
    Calculate minimum image bond length between particles \var{p1} and \var{p2}
   \item
\begin{code}
bond\_angle <p1> <p2> <p3> [type]
\end{code}
Calculate bond angle between particles \var{p1}, \var{p2} and
\var{p3}. If \var{type} is "r" the return value is in radiant. If it
is "d" the return value is in degree. The default for \var{type} is
"r".
   \item
\begin{code}
bond\_dihedral <p1> <p2> <p3> <p4> [type]
\end{code}
Calculate bond dihedral between particles \var{p1}, \var{p2}, \var{p3}
and \var{p4} If \var{type} is "r" the return value is in radiant. If
it is "d" the return value is in degree The default for \var{type} is
"r".
   \item
\begin{code}
part\_at\_dist <p> <dist>
\end{code}
return position of a new particle at distance \var{dist} from \var{p}
with random orientation
   \item
\begin{code}
part\_at\_angle <p1> <p2> <phi> [len]
\end{code}
return position of a new particle at distance \var{len} (default=1.0)
from \var{p2} which builds a bond angle \var{phi} for (\var{p1},
\var{p2}, p-new)
   \item
\begin{code}
part\_at\_dihedral <p1> <p2> <p3> <theta> [phi] [len]
\end{code}
return position of a new particle at distance \var{len} (default=1.0)
from \var{p3} which builds a bond angle \var{phi} (default=random) for
(\var{p2}, \var{p3}, p-new) and a dihedral angle \var{theta} for
(\var{p1}, \var{p2}, \var{p3}, p-new)
  \end{itemize}
 \item
  INTERACTION RELATED
  
  Help functions related to interactions implemented in \es.
  \begin{itemize}
   \item 
     \begin{code}
       calc_lj_shift <lj_sigma> <lj_cutoff>
     \end{code}
     returns the value needed to shift the Lennard Jones potential to zero at the cutoff.

  \end{itemize}

 \item
  VECTOR OPERATIONS

  A vector \var{v} is a tcl list of numbers with an arbitrary length
  Some functions are provided only for three dimensional vectors.
  corresponding functions contain 3d at the end of the name.
  \begin{itemize}
   \item
\begin{code}
veclen <v>
\end{code}
    return the length of a vector
   \item
\begin{code}
veclensqr <v>
\end{code}
    return the length of a vector squared
   \item
\begin{code}
vecadd <a> <b>
\end{code}
    add vector \var{a} to vector \var{b}: return = (\var{a}+\var{b})
   \item
\begin{code}
vecsub <a> <b>
\end{code}
    subtract vector \var{b} from vector \var{a}: return = (\var{a}-\var{b})
   \item
\begin{code}
vecscale <s> <v>
\end{code}
    scale vector \var{v} with factor \var{s}: return = (\var{s}*\var{v})
   \item
\begin{code}
vecdot\_product <a> <b>
\end{code}
    calculate dot product of vectors \var{a} and \var{b}: return = (\var{a}.\var{b})
   \item
\begin{code}
veccross\_product3d <a> <b>
\end{code}
calculate the cross product of vectors \var{a} and \var{b}: return =
(\var{a} x \var{b})
   \item
\begin{code}
vecnorm <v> [len]
\end{code}
    normalize a vector to length \var{len} (default 1.0)
   \item
\begin{code}
unitvec <p1> <p2>
\end{code}
    return unit vector pointing from position \var{p1} to position \var{p2}
   \item
\begin{code}
orthovec3d <v> [len]
\end{code}
return orthogonal vector to \var{v} with length \var{len} (default
1.0) This vector does not have a random orientation in the plane
perpendicular to \var{v}
   \item
\begin{code}
create\_dihedral\_vec <v1> <v2> <theta> [phi] [len]
\end{code}
create last vector of a dihedral (\var{v1}, \var{v2}, res) with
dihedral angle \var{theta} and bond angle (\var{v2}, res) \var{phi}
and length \var{len} (default 1.0). If \var{phi} is ommited or set to
rnd then \var{phi} is assigned a random value between 0 and 2 Pi.
  \end{itemize}
 \item
  TCL LIST OPERATIONS
  \begin{itemize}
   \item
\begin{code}
average <list>
\end{code}
    Returns the avarage of the provided \var{list}
   \item
\begin{code}
list\_add\_value <list> <val>
\end{code}
    Add \var{val} to each element of \var{list}
   \item
\begin{code}
flatten <list>
\end{code}
    flattens a nested \var{list}
   \item
\begin{code}
list\_contains <list> <val>
\end{code}
Checks wether \var{list} contains \var{val}. returns the number of
occurences of \var{val} in \var{list}.
  \end{itemize}
 \item
  REGRESSION
  \begin{itemize}
   \item
\begin{code}
LinRegression <l>
\end{code}
where \var{l} is a listof pairs of points \verb!{ {$x1 $y1} {$x2 $y2} ...} !. 
\verb!LinRegression! returns the least-square linear fit $ax+b$ and 
the standard errors $\sigma_a$ and $\sigma_b$.
   \item
\begin{code}
LinRegressionWithSigma <l>
\end{code}
where \var{l} is a list of lists of points in the form \verb!{{$x1 $y1 $s1} {$x2 $y2 $s2} ...}! 
where \verb!s! is the standard deviation of \verb!y!. 
\verb!LinRegressionWithSigma! returns the least-square linear fit 
$ax+b$, the standard errors $\sigma_a$ and $\sigma_b$, 
covariance $\mathrm{cov}(a,b)$ and $\chi$.
\end{itemize}
\end{itemize}

\subsection{\texttt{t\_random}}
\label{ssec:trandom}
\begin{itemize}
 \item
   Without further arguments,
\begin{code}
t\_random
\end{code}
returns a random double between 0 and 1 using the 'ran1' random number
generator from Numerical Recipes.
 \item
\begin{code}
t\_random int <n>
\end{code}
  returns a random integer between 0 and n-1.
 \item
\begin{code}
t\_random seed
\end{code}
returns a tcl-list with the seeds of the random number generators on
each of the 'n\_nodes' nodes, while
\begin{code}
t\_random seed <seed(0)> ... <seed(n\_nodes-1)>
\end{code}
sets those seeds to the new values respectively, re-initialising the
random number generators on each node.  Note that this is
automatically done on invoking Espresso, however due to that your
simulation will always start with the same random sequence on any node
unless you use this tcl-command to reset the sequences' seeds.
\item Since internally the random number generators' random sequences
  are not based on mere seeds but rather on whole random number
  tables, to recover the exact state of the random number generators
  at a given time during the simulation run (e. g. for saving a
  checkpoint) requires knowledge of all these values. They can be
  accessed by
\begin{code}
t\_random stat
\end{code}
which returns a tcl-list with all status informations for any node (e.
g. 8 nodes $=>$ approx. 350 parameters). To overwrite those internally
in Espresso (e. g. upon restoring a checkpoint) submit the whole list
back using
\begin{code}
t\_random stat <status-list>
\end{code}
with \var{status-list} being the tcl-list mentioned above without any
braces.  Be careful! A complete recovery of the current state of the
simulation is only possible if you make sure to include a call to the
\texttt{sort_particles} command after you saved the blockfile to make
sure random numbers are applied in the same order.
\end{itemize}
The C implementation is t\_random

\subsection{\texttt{The bit\_random command}}
\label{ssec:bitrandom}
\begin{itemize}
 \item
  Without further arguments,
\begin{code}
bit\_random
\end{code}
returns a random double between 0 and 1 using the R250 generator
XOR-ing a table of 250 linear independent integers.
 \item
\begin{code}
bit\_random seed
\end{code}
returns a tcl-list with the seeds of the random number generators on
each of the 'n\_nodes' nodes, while
\begin{code}
bit\_random seed <seed(0)> ... <seed(n\_nodes-1)>
\end{code}
sets those seeds to the new values respectively, re-initialising the
random number generators on each node.  Note that this is
automatically done on invoking Espresso, however due to that your
simulation will always start with the same random sequence on any node
unless you use this tcl-command to reset the sequences' seeds.
\item Since internally the random number generators' random sequences
  are not based on mere seeds but an array of 250 linear independent
  integers whose bits are used as matrix elements which are XOR-ed, to
  recover the exact state of the random number generators at a given
  time during the simulation run (e. g. for saving a checkpoint)
  requires knowledge of all these values. They can be accessed by
\begin{code}
bit\_random stat
\end{code}
which returns a tcl-list with all status informations for any node (e.
g. 8 nodes $=>$ approx. 2016 parameters). To overwrite those
internally in Espresso (e. g. upon restoring a checkpoint) submit the
whole list back using
\begin{code}
bit\_random stat <status-list>
\end{code}
with <status-list> being the tcl-list mentioned above without any
braces.  Be careful! See \ref{ssec:trandom} for more information
on how to recover of the current state, include the sequence the random
numbers are applied.
\item Note further that the bit-wise display of integers, as it is
  used by this random number generator, is platform dependent. As long
  as you stay on the same architecture this doesn't matter at all;
  however, it wouldn't be wise to use a checkpoint including the state
  of the R250 to restart the simulation on a different platform - most
  likely, the integers will have a different bit-muster leading to a
  completely different random matrix.  So, if you're using this random
  number generator, always remain on the same platform!
\end{itemize}


\section{Checking for features of \es}

In an \es-Tcl-script, you can get information whether or not one or
some of the features are compiled into the current program with help
of the following Tcl-commands:
\begin{itemize}
 \item
\begin{code}
code\_info
\end{code}
provides information on the version, compilation status and the debug
status of the used code. It is highly recommended to store this
information with your simulation data in order to maintain the
reproducibility of your results.  Exemplaric output:
\begin{tclcode}
ESPRESSO: v1.5.Beta (Neelix), Last Change: 23.01.2004
{ Compilation status { PARTIAL_PERIODIC } { ELECTROSTATICS }
{ EXTERNAL_FORCES } { CONSTRAINTS } { TABULATED }
{ LENNARD_JONES } { BOND_ANGLE_COSINE } }
{ Debug status { MPI_CORE FORCE_CORE } }
\end{tclcode}
 \item
\begin{code}
has\_feature <feature> ...
\end{code}
tests, if \var{feature} is compiled into the \es{} kernel. A list of
possible features and their names can be found here.
\item
\begin{code}
  require\_feature <feature> ...
\end{code}
tests, if \var{feature} is feature is compiled into the \es{} kernel,
will exit the script if it isn't and return the error code 42. A list
of possible features and their names can be found here.
\end{itemize}


%%% Local Variables: 
%%% mode: latex
%%% TeX-master: "ug"
%%% End: 
