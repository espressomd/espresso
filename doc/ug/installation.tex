%\chapter{Installation}
%\label{chap:install}

\begin{itemize}
\item Compiling \es{} is a necessary evil
\item Features can be compiled in or not
\item For maximal efficiency, compile in only the features that you
  use
\item \es{} can be obtained from the \es{} home page
  \footnote{\url{http://www.espresso.mpg.de}}.
\end{itemize}



\section{Requirements}
\label{sec:requirements}

\begin{description}
\item[Tcl/Tk] \es{} requires the Toolkit Command Language Tcl/Tk
  \footnote{\url{http://www.tcl.tk/}} in the version 8.3 or later.
  Some example scripts will only work with Tcl 8.4. You do not only
  need the interpreter, but also the header files and libraries.
  Depending on the operating system, these may come in separate
  development packages. If you want to use a graphical user interface
  (GUI) for your simulation scripts, you will also need Tk.
  
\item[FFTW] In addition, \es{} needs the FFTW library
  \footnote{\url{http://www.fftw.org/}} for Fourier transforms.
  ESPResSo can work with both the 2.1.x and 3.0.x series. Again, the
  header files are required.
  
\item[MPI] Finally, if you want to use ESPResSo in parallel, you need
  a working MPI environment (version 1.2). ESPResSo currently supports
  the following MPI implementations:
  \begin{itemize}
  \item LAM/MPI is the preferred variant
  \item MPICH, which seems to be considerably slower than LAM/MPI in
    our benchmarks.
  \item On AIX systems, \es{} can also use the native POE parallel
    environment.
  \item On DEC/Compaq/HP OSF/Tru64, \es{} can also use the native
    dmpirun MPI environment.
  \end{itemize}
\end{description}

\section{Quick start}

In many cases, to compile \es{}, it is enough to execute the following
sequence of two steps in the directory where you have unpacked the
sources:

\begin{verbatim}
> configure
> make
\end{verbatim}

In some cases, \eg{} when \es{} needs to be compiled for several
different machines or when different versions with different features
are required, it might be useful to execute the commands not in the
source directory itself, but to start \texttt{configure} from another
directory (see section \vref{sec:builddir}). Furthermore, many
features of \es{} can be selectively turned on or off in the local
configuration header of \es{} (see section \vref{sec:myconfig}) before
starting the compilation with \texttt{make}.

The shell script \texttt{configure} prepares the source code for
compilation. It will determine how to use and where to find the
different libraries and tools required by the compilation process, and
it will test what compiler flags are to be used.  The script will find
out most of these things automatically.  If something is missing, it
will complain and give hints how to solve the problem.  The
configuration process can be controlled with the help of a number of
options that are explained in section \vref{sec:configure}.

The command \texttt{make} will compile the source code. Depending on
the options passed to the program, \texttt{make} can also be used for
a number of other things:
\begin{itemize}
\item It can install and uninstall the program to some other
  directories. However, normally it is not necessary to actually
  \textit{install} \es{} to run it.
\item It can test the \es{} program for correctness.
\item It can build the documentation.
\end{itemize}
The details of the usage of \texttt{make} are described in section
\vref{sec:make}.

When these steps have successfully completed, \es{} can be started
with the command (see section \vref{sec:run})
\begin{verbatim}
> Espresso
\end{verbatim}

\section{Separation of source and build directory}
\label{sec:builddir}

\begin{itemize}
\item srcdir
\item builddir
\item when called from srcdir, builddir will be
  created automatically (obj-PLATFORM)
\item in srcdir, you can do 
  \begin{itemize}
  \item configure
  \item make
  \item Espresso 
  \end{itemize}
  but all files will be created in builddir (obj-PLATFORM)!
\end{itemize}

\section{The configuration header \texttt{myconfig.h}}
\label{sec:myconfig}

\verb!(1) > configure --with-myconfig=MYCONFIG_HEADER!\\
or\\
\verb!(2) > make myconfig=MYCONFIG_HEADER!

\noindent 
Sample file: \verb!$srcdir/myconfig-sample.h! \\
Default file: \verb!$builddir/config/myconfig.h!

Priority: 
\begin{enumerate}
\item builddir
\item srcdir
\item builddir/config (default, don't touch)
\end{enumerate}

\section{Running configure}
\label{sec:configure}

\texttt{configure} will save the assembled information in the
different \texttt{Makefile}s, the header file \texttt{acconfig.h}, and
some other shell scripts required later on.

The following options are recognised by \texttt{configure}:
\begin{verbatim}
--prefix=PREFIX (Default: $HOME/Espresso)
--exec-prefix=EXEC_PREFIX (DEFAULT: PREFIX)

--enable-chooser (Default: depends on current directory)
--with-myconfig=MYCONFIG_HEADER (Default: myconfig.h)
--enable-config=KNOWN_CONFIG (Default: no)

--enable-debug (Default: off)
--enable-profiling (Default: off)
--disable-processor-optimization (Default: enabled)
--enable-xlc-qipa (Default: yes)

--with-mpi=MPI (Default: guess)
--with-efence (Default: no)
--with-tcl=VERSION (Default: guess)
--with-tk[=VERSION] (Default: no)
--with-fftw=VERSION (Default: guess)
\end{verbatim}

\section{Compiling, testing and installing \es}
\label{sec:make}

\begin{verbatim}
> make [all] [myconfig=MYCONFIG_HEADER]

  Compiles Espresso. 

  Variables:
  ----------
  myconfig=MYCONFIG_HEADER
    Sets the name of the local configuration header file where you can
    turn on the different features and debug messages.
    This can be useful if you want to compile Espresso with different
    sets of features from the same sources.
    WARNING:
    If you use this when compiling Espresso, it is necessary to use
    the same setting when you recompile Espresso or to call "make
    clean" before. Otherwise, you will end up with an inconsistent
    version of Espresso.
    DO NOT USE THIS WHEN YOU WORK ON THE SOURCE CODE! 
    USE configure --with-myconfig instead!

> make check [processors=PROC] [tests=TESTS]

  Runs the Espresso testsuite.

  Variables:
  ----------
  processors=PROC (Default: "1 2 3 4 6 8")
    Sets the list of processor numbers that are to be tested,
    e.g. 'make check processors="1 2"' will run the testsuite on only
    one and two processors. 

  tests=TESTS (Default: all tests)
    Sets the tests of the testsuite that are run, e.g. 
    'make check tests="madelung.tcl"' will run only the test
    madelung.tcl.

> make clean

  Deletes the files that were created during the compilation.

> make mostlyclean

  Deletes most files that were created during the compilation. Will
  keep for example the built doxygen documentation and the Espresso
  binary.

> make dist [internal=1]

  Creates a .tar.gz-file of the Espresso sources. This will include all
  source files of Espresso as they currently are in the source
  directory, i.e. it will include local changes.
  This is useful to give your version of Espresso to other people.

  Variables:
  ----------
  internal=1
    When this is give, include the "internal" directory into the
    distribution file.

> make dist-internal

  Same as "make dist internal=1"

> make install [prefix=DIR] [exec-prefix=DIR]

  Installs Espresso.

  Variables:
  ----------
  prefix=PREFIX (Default: configure option --prefix)
    Sets the directory where to install Espresso. This
    defaults to the directory given to configure.

  exec-prefix=PREFIX (Default: configure option --exec-prefix)
    Sets the directory where to install the executable files of
    Espresso. This is only required, when the executable files are to
    be installed into some architecture-specific directory. Otherwise,
    it is identical to the prefix.

> make uninstall [prefix=DIR] [exec-prefix=DIR]

  Uninstalls Espresso. The variables are identical to the variables of
  "make install".
\end{verbatim}

\section{Running \es}
\label{sec:run}

A number of wrapper scripts are used in running \es{}:
\begin{itemize}
\item The script \texttt{Espresso} in the source and build directory
  will try to run the compiled version of \es. If it is called from
  the source directory, it assumes that \es{} was also configured in
  the source directory and will try to recursively start the script in
  the corresponding \texttt{obj-PLATFORM} build directory. If it is
  called in the build directory, it will start the \es-binary with the
  right MPI implementation.
\item The chooser script \texttt{Espresso} 
  \begin{itemize}
  \item installed when \verb!--enable-chooser! was given
  \item installed to bindir
  \item tries to run the correct version of the MPI-wrapper
    \texttt{Espresso}
  \end{itemize}
\item The MPI-wrapper \texttt{Espresso}
  \begin{itemize}
  \item installed next to \es{} binary
  \item starts the binary with the right MPI implementation
  \end{itemize}
\item The \es{} binary \texttt{Espresso-bin} can also be started
  directly, however, it requires that the environment variable
  \verb!ESPRESSO_SCRIPTS! is set to the directory where the scripts
  are installed (usually \verb!$(prefix)/lib/espresso/scripts! or
  \verb!$(prefix)/share/espresso/scripts!).
\end{itemize}



%%% Local Variables: 
%%% mode: latex
%%% TeX-master: "ug"
%%% End: 
