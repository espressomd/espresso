% Copyright (C) 2010,2012,2013,2014,2015,2016 The ESPResSo project
% Copyright (C) 2002,2003,2004,2005,2006,2007,2008,2009,2010 
%   Max-Planck-Institute for Polymer Research, Theory Group
%  
% This file is part of ESPResSo.
%   
% ESPResSo is free software: you can redistribute it and/or modify it
% under the terms of the GNU General Public License as published by the
% Free Software Foundation, either version 3 of the License, or (at your
% option) any later version.
%  
% ESPResSo is distributed in the hope that it will be useful, but
% WITHOUT ANY WARRANTY; without even the implied warranty of
% MERCHANTABILITY or FITNESS FOR A PARTICULAR PURPOSE.  See the GNU
% General Public License for more details.
%  
% You should have received a copy of the GNU General Public License
% along with this program.  If not, see <http://www.gnu.org/licenses/>.
%
\chapter{Sample scripts}
\label{chap:samples}

In the directory \es{}/samples you find several scripts that can serve
as samples how to use \es{}.
\begin{description}
\item[lj\_liquid.tcl] Simple Lennard-Jones particle liquid. Shows the
  basic features of \es: How to set up system parameters, particles
  and interactions. How to warm up and integrate. How to write
  parameters, configurations and observables to files. How to handle
  the connection to VMD.
%\item[kremerGrest.tcl] This reproduces the data of \citet{kremer90a}:
%  Multiple systems with different number of neutral polymer chains of
%  various lengths are simulated for very long times at melt density
%  0.85 while their static and some dynamic properties are measured.
%  Shows the advanced features of \es{}: How to run several simulations
%  from a single script. How to use online-analysis (The analyze
%  command) with comparision to expectation values. How to get averages
%  of the observables. How to set/restore checkpoints (Using
%  Checkpoints, saving configurations) including auto-detection of
%  previously derived parts of the simulation(s). How to create
%  gnuplots from within the script and combine multiple plots onto
%  duplex pages (Statistical Analysis and Creating Gnuplots).  In the
%  end the script will provide plots of all important quantities as
%  .ps- and .pdf-files while compressing the data-files. Note however,
%  that the simulation uses the original time scale, hence it may take
%  quite some time to finish.
\item[pe\_solution.tcl] Polyelectrolyte solution under poor solvent
  condition. Test case for comparison with data produced by polysim9
  from M.Deserno. Note that the equilibration of this system takes
  roughly $15000 \tau$.
\item[pe\_analyze.tcl] Example for doing the analysis after the actual
  simulation run (offline analysis). Calculates the integrated ion
  distribution $P(r)$ for several different time slaps, compares them
  and presents the final result using gnuplot to generate some
  ps-files.
\item[harmonic\_oscillator.tcl] A chain of harmonic oscillators. This
  is a $T=0$ simulation to test the energy conservation.
\item[espresso\_logo.tcl] The \es-logo, the exploding espresso cup,
  has been created with this script. It is a regular simulation of a
  polyelectrolyte solution. It makes use of some nice features of the
  part command (see section \vref{tcl:part}, namely the capability to
  fix a particle in space and to apply an external force.
\end{description}

%%% Local Variables: 
%%% mode: latex
%%% TeX-master: "ug"
%%% End: 
