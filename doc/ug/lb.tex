\chapter{Lattice Boltzmann}
\label{sec:lb}
For an implicit treatment of a solvent \es allows to couple the molecular dynamics 
part to a Lattice Boltzmann fluid. The Lattice Boltzmann Method is a fast, lattice 
based method that allows in its ``pure'' form to calculate fluid flow in various boundary
conditions also in complex geometries. Coupled to molecular dynamics, it is mainly used to 
take hydrodynamic interactions into account computationally efficiently. The main
goal in \es is, of course the coupling to MD and therefore available geometries and
boundary conditions are somewhat limited compared to ``pure'' codes. 
Here we restrict the documentation on the interface. For a more detailed description


\section{Setting up and LB fluid}
\begin{essyntax}
  lbfluid
  \opt{agrid  \var{agrid}}
  \opt{dens  \var{density}}
  \opt{visc  \var{viscosity}}
  \opt{tau  \var{lb\_timestep}}
  \opt{bulk_visc  \var{bulk\_viscosity}}
  \opt{ext_force  \var{f_x} \var{f_y} \var{f_z}}
  \opt{friction   \var{gamma} } 
  \opt{gamma_odd  \var{gamma\_odd}}
  \opt{gamma_even  \var{gamma\_even}}
  \begin{features}
  \required{LB}
  \end{features}
\end{essyntax}
The \lit{lbfluid} command serves to initialize the fluid with a given set
of parameters. It is also possible to change parameters on the fly, but this will
only rarely be done in practice. Before being able to use the LBM, it is necessary
to set up a box of a desired size. The parameter \lit{agrid} serves to set up
the lattice constant of the fluid, so the size of the box in every direction must be a
multiple of \var{agrid}. 

In \es the LB scheme and the MD scheme are not synchronized: In one
LB time step typically several MD steps are performed. This allows to speed
up the simulations and is adjusted with the parameter \var{tau}.
The parameters \var{dens} and \var{visc} set up the
density and viscosity of the LB fluid in (usual) MD units.
Internally the LB implementation works with a different set of units:
all lengths are expressed in \var{agrid}, all times in \var{tau} and so on.
Therefore changing  \var{agrid} and \var{tau}, might change the behaviour 
of the LB fluid, \eg at boundaries, due to characteristics of the LBM
itself. Currently it is not possible to precisely give a parameter set
where reliable results are expected, but we are currently performing 
a study on that. Therefore the LBM should \emph{not be used as a black box}, but
only after a careful check of all parameters that were applied. Again: 
You have to \emph{know what you are doing} before you can use the Lattice-Boltzmann
module productively. 

The parameter \lit{ext_force} allows to apply an external body force density
that is homogenous over the fluid. It is again to be given in MD units.
The parameter \lit{bulk_viscosity} allows to tune the bulk viscosity of the fluid
and is given in MD units. In the limit of low Mach (often also low Reynolds) number
the results should be independent of the bulk viscosity. 
\lit{gamma_odd} and \lit{gamma_even} are the relaxation parameters for the kinetic
modes. Due to their somewhat obscure nature they are to be given directly in 
LB units.

Before running a simulation at least the following parameters must be set up:
\lit{agrid}, \lit {dens}, \lit{visc}, \lit{tau}. For the other parameters, the following
defaults are taken:  \var{bulk\_viscosity}=0, \var{gamma}=0, \var{gamma\_odd}=0, \var{gamma\_even}=0.

\section{LB as a thermostat}

\section{Reading and setting single lattice nodes}
\begin{essyntax}
  lbnode \var{x} \var{y} \var{z} \alt{print \asep set} \var{args}
  \begin{features}
  \required{LB}
  \end{features}
\end{essyntax}
The \lit{lbnode} command allows to inspect (\lit{print}) and (\lit{modify}) single LB nodes. 
Note that the indexing in every direction starts with 0. 
For both commands you have to specify what quantity should be printed
or modified. Print allows the following arguments: \\
\begin{tabular}{p{0.2\columnwidth}p{0.5\columnwidth}}
  \lit{rho}\ & the density (scalar). \\
  \lit{u} & the fluid velocity (three floats: $u_x$, $u_y$, $u_z$) \\
  \lit{pi} & the fluid velocity (six floats: $\Pi_{xx}$, $\Pi_{xy}$, $\Pi_{yy}$, $\Pi_{xz}$,  $\Pi_{yz}$,  $\Pi_{zz}$) \\
  \lit{pi_neq} & the nonequilbrium part of the pressure tensor, components as above. \\
  \lit{pop} & the 19 population (check the order from the source code please).
\end{tabular} \\
Example:
The line
\begin{tclcode}
puts [ lbnode 0 0 0 print u ]
\end{tclcode}
prints the fluid velocity in node 0 0 0 to the screen.
The command \lit{set} allows to change the density or fluid velocity in a single node. Setting
the other quantities can easily be implemented.
Example:
\begin{tclcode}
puts [ lbnode 0 0 0 set u 0.01 0. 0.]
\end{tclcode}

\section{Setting up boundary conditions}
\begin{essyntax}
  lbboundary \var{shape} \var{shape\_args} 
  \begin{features}
  \required{LB_BOUNDARIES}
  \end{features}
\end{essyntax}
The \lit{lbboundary} command syntax is very close to the \lit{constraint} syntax, as 
usually one wants the hydrodynamic boundary conditions to be shaped similar to 
the MD boundaries. Currently the above mentioned shapes are available and their syntax follows exactly
the syntax of the constraint command. For example
\begin{tclcode}
  lbboundary wall dist 1.5 normal 1. 0. 0. 
\end{tclcode}
creates a planar boundary condition at distance 1.5 from the origin of the coordinate system
where the half space $x>1.5$ is treated as normal LB fluid, and the other half space 
is filled with boundary nodes.

Intersecting boundaries are in principle possible but must be treated with care. In the current,
only partly satisfactory implementation for every LB node the closest boundary is determined
and in case of a negative distance to that boundary (meaning being inside the wall) lets \es decide that
it is a wall node, and a positive one, that it is not.
This allows example multiple parallel channels, but is is not always clear that a rectangularly shaped
channel can be constructed. Suggestions on how to solve this issue are very welcome.

Currently, only the so called ``link-bounce-back'' algorithm for wall nodes is available. This creates a boundary that
is located approximately midway between the lattice nodes, so in the above example this corresponds indeed to 
a boundary at $x=1.5$. Note that the location of the boundary is unfortunately not independent of the 
viscosity. This can \eg be seen when using the sample script \lit{poisseuille.tcl}
with a high viscosity.
