\documentclass[aip,jcp,reprint,a4paper,onecolumn,nofootinbib,amsmath,amssymb]{revtex4-1}
% Options for onecolumn and a4paper
\linespread{1.0}
\usepackage[expansion,protrusion,tracking=smallcaps]{microtype}
\usepackage{chemformula,tikz}
\usepackage{listings,xcolor}
\lstset{
  basicstyle = \small\ttfamily,
  keywordstyle = \color{blue!80!black},
  commentstyle = \color{green!40!black},
  columns = fullflexible,
  xleftmargin = 1em
}
\lstdefinelanguage{espresso}[]{TCL}{%
  morekeywords={part,reaction,minimize_energy},
  deletekeywords={range}
}
\lstnewenvironment{espresso}[1][]%
  {\lstset{language=espresso,#1}}%
  {}
\lstnewenvironment{bash}[1][]%
  {\lstset{keywordstyle = \color{red!80!blue},keywords={\$},#1}}%
  {}
\usepackage{xspace}
\newcommand\code{\lstinline}
\newcommand{\es}{\mbox{\textsf{ESPResSo}}\xspace}
\newcommand\codees{\lstinline[language=espresso]}
\renewcommand\thefootnote{\fnsymbol{footnote}}
\begin{document}
\title{Catalytic Reactions Tutorial}
\author{Henri Menke}
\email[]{\texttt{henri@icp.uni-stuttgart.de}}
\affiliation{Institute for Computational Physics,
  Universit\"at Stuttgart,
  Allmandring 3,
  D-70569 Stuttgart,
  Germany}
\author{Joost de Graaf}
\email[]{\texttt{jgraaf@icp.uni-stuttgart.de}}
\affiliation{Institute for Computational Physics,
  Universit\"at Stuttgart,
  Allmandring 3,
  D-70569 Stuttgart,
  Germany}
\affiliation{School of Physics and Astronomy,
  University of Edinburgh,
  Scotland, Edinburgh EH9 3JL,
  United Kingdom}
\date{\today}

\begin{abstract}
  In this tutorial we explore how to simulation catalytic reactions
  using \es.  We discuss the reaction scheme implemented in \es\ and
  use one of them to study the enhancement of translational diffusion
  of a colloid.  This should give a basic introduction on how to setup
  simulations involving catalytic reaction in \es\ and ideally enable
  you to conduct your own numerical experiments.
\end{abstract}

\maketitle

\section{Cataltic Reactions}

There are processes in nature which do not occur at reasonable
temperatures because the energy penalty is too high to be overcome by
mere fluctuations.  The activation energy of such processes can in
some cases be lowered significantly in the presence of a catalyzer.

We may write a general reaction as
\begin{equation}
  \label{eq:equi}
  \ch{!(reactants)(R) <=> !(products)(P)}
\end{equation}
where \ch{R} are the reactants and \ch{P} are the products.  The
double headed arrow suggests that this reaction is in an equilibrium.
The equilibrium reation~\eqref{eq:equi} can be described by the
equilbrium reaction constant
\begin{equation}
  \label{eq:const}
  K_{\text{eq}} = \frac{k_{\text{eq},+}}{k_{\text{eq},-}} = \frac{[\ch{P}]}{[\ch{R}]}
\end{equation}
with $[\ch{P}]$ and $[\ch{R}]$ the product and reactant concentration
and $k_{\textrm{eq},\pm}$ the forward and backward reaction rates.
The chief effect of the reaction is a change of the concentrations of
reactants and products over time.  We find the following coupled
differential equations
\begin{subequations}
  \label{eq:diff}
  \begin{align}
    \label{eq:diff-equi-a}
    \frac{d[\ch{R}]}{dt} &= + k_{\text{eq},-}[\ch{P}] - k_{\text{eq},+}[\ch{R}] \;, \\
    \label{eq:diff-equi-b}
    \frac{d[\ch{P}]}{dt} &= - k_{\text{eq},-}[\ch{P}] + k_{\text{eq},+}[\ch{R}]  \;.
  \end{align}
\end{subequations}

A catalyzer shifts the free energy landscape in favour of one of the
constituents \ch{R} or \ch{P}.  We choose \ch{P}, since it stands for
the products and we want to enhance the forward reaction.
\begin{equation}
  \ch{!(reactants)(R) <=>>[ !(catalyzers)(C) ] !(products)(P)} \;.
\end{equation}
If the back reaction rate is tiny it eventually becomes negligible and
the reaction only occurs in one direction.
\begin{equation}
  \label{eq:cat}
  \ch{!(reactants)(R) ->[ !(catalyzers)(C) ] !(products)(P)} \;.
\end{equation}
In the case of excessive catalysis~\eqref{eq:cat} the back reaction
rate $k_{\text{eq},-}$ can be set to zero and the forward reaction
rate is enhanced $k_{\text{eq},+} \to k_{\text{ct}}$.  The rate
equations~\eqref{eq:diff} then reduce to
\begin{equation}
  \label{eq:diff-cat}
  -\frac{d[\ch{R}]}{dt} = \frac{d[\ch{P}]}{dt} = k_{\text{ct}}[\ch{R}] \;.
\end{equation}

What we have discussed so far is a continuum model, as no actual
particles are involved in the picture, only concentrations.  To
simulate such a simple catalytic reaction with molecular dynamics we
need to come up with a minimal model.  Phenomenologically, catalysis
is triggered by a contact interaction, i.e.~the reactant has to touch
the catalyzer.  \es{} only has point particles which means that
contact will happen rarely.  To compensate for this, we define a
\emph{reaction range} around the catalyzer inside which the conversion
of a reactant to a product is feasible.  Next we have to think about
how to implement \eqref{eq:diff-cat} is this picture.  Simply
integration \eqref{eq:diff-cat} yields an exponential decay as a
function of time.  The reaction move is a stochastic process and takes
place with a certain probability, depending on the reaction rate.
This probability is inversely proportional to the exponential decay.
We assign
\begin{equation}
  \label{eq:prob}
  P_{\text{move}} = 1 - \mathrm{e}^{-k\,\Delta t}
\end{equation}
with the reaction rate $k \in \{ k_{\text{eq},+}, k_{\text{eq},-},
k_{\text{ct}} \}$ and the simulation time step $\Delta t$.

\section{Cataltic Reactions in \es}

There are two implementations of the reaction
scheme~\eqref{eq:diff-equi-a},~\eqref{eq:diff-equi-b},
and~\eqref{eq:diff-cat}, which we will refer to as \emph{number
  conserving} and \emph{non-number-conserving}.  In the following we
will find out what this actually means.

\subsection{Non-Number-Conserving Scheme}

In \es\ we assume the equilibrium constants in
equations~\eqref{eq:diff} to be equal, i.e.~$k_{\text{eq},+} =
k_{\text{eq},-}$.  Thus, according to~\eqref{eq:const}, the reaction
constant $K = 1$.  The equilibrium constant is applied to all the
particles in the system and determines the conversion rate between the
reactants and products.  For reactants in the vicinity of the
catalyzer the forward reaction is favoured and no back conversion will
take place there.  Also the forward reaction rate is enhanced to
$k_{\text{ct}}$ and can be specified as an input parameter.

\begin{figure}
  \centering
  \includegraphics{FIGURES/non-number-conserving}
  \caption{Illustration of the \emph{non-number-conserving} scheme.
    Within a cut-off radius $r$ of the catalyzer (green), the reactant
    particles (blue) can be converted into products (red); inert
    particles (white) are left unaffected by the reactions taking
    place. The catalytic conversion is indicated using the arrow
    labeled $k_{\text{ct}}$, which shows the result of a conversion
    event taking place in one time step $\Delta t$ (from left panel to
    right panel). Outside of this region a homogeneous equilibrium
    reaction takes place, that pushes the out-of-equilibrium
    concentrations produced by the catalyzer back towards
    equilibrium. This is indicated by the two arrows which cause
    conversion of products into reactants with $k_{\text{eq},-}$ and
    reactants into products with $k_{\text{eq},+}$.  Even though
    $k_{\text{eq},+} = k_{\text{eq},-} = k_{\text{eq}}$ in \es, we
    show them seperately to emphasise that one is a forward and one is
    a back reaction.}
  \label{fig:nnc}
\end{figure}

Figure~\ref{fig:nnc} shows the reaction scheme explained before
pictorially.  Within the reaction range, determined by $r$, the
equilibrium reactions are also allowed to take place, but additionally
the forward reaction is catalysed.  Within one time step bulk
reactions take place with the equilibrium constant.  These might occur
in both directions and push the the out-of-equilibrium concentrations
produced by the catalyzer back towards equilibrium.  From this we
already see, that it does not make sense to set $k_{\text{eq}} >
k_{\text{ct}}$.  This would basically shadow the effect of the
catalyzer in contrast to the equilibrium reactions in bulk.

By looking at the sketch in figure~\ref{fig:nnc} and carefully
following the discussion of the scheme above, we may now have
understood why the scheme is called \emph{non-number-conserving}.  The
problem is, that forward and backward reaction are not balanced.  Even
though the overall number of particles is conserved the number of
particles per species is not.  This is especially peculiar, if the any
of the species swapped around are charged, because the electrostatic
algorithms break down for non-neutral boxes.  Not only that
``teleporting'' charged particles disturbes the long-range
interactions, but charge conservation is violated as well.  This is
evident from the sketch, where we have two reactants and one product
before the reaction and one reactant and two products after the
reaction.  This issue can be rectified in the \emph{number-conserving}
scheme.

\subsection{Number-Conserving Scheme}

The aforementioned number conservation issues can be addressed in the
following way.  We disallow bulk reactions of single
particles\footnote{We noticed that we set the equilibrium reaction
  rate $k_{\text{eq}} = 0$ in most cases anyway.} and allow a reaction
inside the reaction range involving only rectant-product pairs.
Simply exchanging the particles in the pair would not yield a large
effect, which is why we introduce an additional notion of symmetry
breaking.  Consider a Janus particles which is coated with \ch{Pt} on
one hemisphere and with \ch{Au} on the other hemisphere.  Both
surfaces enhance a reaction, here\cite{Gibbs_10,Wheat_10}
\begin{align}
  \label{eq:H2O2}
  \ch{
    H2O2 &->[ Pt ] 2 H^+ + 2 e^- + O2 \\
    2 H^+ + 2 e^- + H2O2 &->[ Au ] 2 H2O
  }
\end{align}
This reaction is depicted in figure~\ref{fig:janus}.  The large arrow in
the sketch points in the direction of movement induced by the
reaction.  We refer to upper and lower hemisphere or half-space in the
following, where upper and lower are to be seen with respect to the
orientation of the particle.  In the present case one might choose
\ch{Au} as the upper hemisphere and \ch{Pt} as the lower hemisphere.

\begin{figure}
  \centering
  \includegraphics{FIGURES/janus-particle}
  \caption{A Janus particle, coated with a catalytic surface on both
    hemispheres may support two different reactions.  This results in
    breaking of rotational and reflection symmetry which propells the
    particle in the direction of the large arrow.}
  \label{fig:janus}
\end{figure}

For simplicity in the simulation we abstract away all the reaction
mechanism including the electron transport through the Janus particle
and the proton transport around it.  We end up with a simple scheme,
depicted in figure~\ref{fig:nc}.

As already explained, reactions can only take place for
reactant-product pairs.  The conversion is such that a reactant is
converted to a product and a product is converted to a reactant.  This
obviously ensures conservation of the particle number of each type,
respectively.  To model a Janus particle as in figure~\ref{fig:janus} we
allow the exchange move to only take place for a chosen configuration.
For the pair to be eligible for the reaction move, the following
conditions have to be met:
\begin{enumerate}
\item Both partners of the reactant-product pair have to reside within
  the reaction range.
\item The product has to reside in the upper half-space of the
  reaction range..
\item The reactant has to reside in the lower half-space of the
  reaction range.
\end{enumerate}
This is described pictorially in figure~\ref{fig:nc}.

\begin{figure}
  \centering
  \includegraphics{FIGURES/number-conserving}
  \caption{Illustration of the \emph{number-conserving} scheme.  The
    cut-off range of $r$ around the catalzer (green) is subdivided
    into two half-spaces in analogy to figure~\ref{fig:janus}.  A
    reactant-product pair (blue and red connected by dotted line) can
    be converted by swapping them around if the product (red) resided
    in the upper half-space (silver background) and the corresponding
    reactant (blue) resides in the lower half-space (gold background).
    The catalytic reaction leads to a conversion of one
    reactant-product pair of particles in this example, denoted by the
    arrows annotated with $k_{\text{ct}}$.  We can see, that only a
    well-positioned pair is converted.  The second reactant-product
    pair within the reaction range is not viable for conversion.
    Additionally, there are no bulk reactions.}
  \label{fig:nc}
\end{figure}

\subsection{Usage in \es}

Catalytic reactions can be enabled in \es\ by compiling in the
eponymous feature \code{CATALYTIC_REACTIONS}.  The number-conserving
method additionally requires the \code{ROTATION} feature to give the
particles an orientation vector.  In \es{} the orientation of the
particle is defined by a quaternion; this in turn defines a rotation
matrix that acts on the particle's initial orientation (along the
z-axis), which then defines the particles current
orientation~\cite{UG,Limbach_06,Arnold_13}.

In \es\ particles are set up using the \codees{part} command.  It
allows to set various options, such as the initial position
(mandatory), the type or the charge.  To setup the reaction there is
the \es\ command \codees{reaction}, which operates on particle types.
The general syntax is
\begin{espresso}
reaction reactant_type R catalyzer_type C product_type P range r ct_rate k_ct
    [eq_rate k_eq] [react_once on/off] [swap on/off]
\end{espresso}
The parameters in square brackets are optional and can be left out.
They will then assume their default values which are \codees{eq_rate 0.0},
\codees{react_once off}, and \codees{swap off}.

\noindent\textbf{Important:} Due to the method of implementation there
can only be one reaction.  You can alter the reaction parameters, but
you may not change the reaction partners.

To set up a reaction between the types 1 and 2, where particles of
type 3 act as catalyzers with a reaction range of 1.5 around them with
a reaction rate of 20, one types
\begin{espresso}
reaction reactant_type 1 catalyzer_type 3 product_type 2 range 1.5 ct_rate 20
\end{espresso}
Here we have left out the optional parameters, but their meaning is
nevertheless important.  The first one, \codees{eq_rate}, should be
self explanatory; it sets the equilibrium reaction rate as detailed in
the introductory sections.  The \codees{react_once} parameter
determines whether a particle can take part in a only single reaction
move per time step.  In the case of multiple catalyzers in the system
a particle might be tagged for reaction several times by different
catalyzers because their reaction ranges overlap and the reactant is
inside this overlap.  This can be prevented by setting
\code{react_once on}.  That way the reaction rate is independent of
the density of catalyzers.  Finally, the parameter \codees{swap}
determines which scheme to use; \codees{swap off} uses the not number
conserving scheme, \codees{swap on} uses the number-conserving scheme.
The name `swap' originates from the sketch figure~\ref{fig:nc}, because
it looks as if the particles are swapped.

However, the \codees{swap} option does not really swap the particles,
but only exchanges their type (and their charge, if \es\ was compiled
with \code{ELECTROSTATICS}).

\section{Configuring \es\ for Catalytic Reactions}

For this tutorial to work you need a version of \es\ containing the
catalytic reactions feature with the \codees{swap} mechanism.
Therefore check out the \es\ online repository at
\url{https://github.com/espressomd/espresso}.  If you have installed
\code{git}, you can issue on the command line
\begin{bash}
$ git clone https://github.com/espressomd/espresso.git
\end{bash}
Now you are ready to build \es.  Change to the newly created directory
and use the following commands to configure \es\ for compilation.
\begin{bash}
$ mkdir build
$ cd build
$ cmake ..
\end{bash}
After this, you will need to copy the \code{myconfig-sample.hpp} file
into \code{myconfig.hpp} and select the appropriate \code{FEATURES} in
the latter.
\begin{bash}
$ cp myconfig-sample.hpp myconfig.hpp
\end{bash}
To run all the tutorials you need to uncomment the following \code{FEATURES}:
\begin{lstlisting}[language=c]
#define ROTATION
#define ROTATIONAL_INERTIA
#define LANGEVIN_PER_PARTICLE
#define CATALYTIC_REACTIONS
#define LENNARD_JONES
\end{lstlisting}
Now you are ready to build \es.
\begin{bash}
$ make -j
\end{bash}
Next you can unpack the archive with the tutorial files in this
directory. You will find two folders, one called `EXERCISES' and one
called `SOLUTIONS'.


\section{The Enhanced-Diffusion Tutorial}

In the folder `EXERCISES' you will find the \code{reaction.tcl} file.
It is a tutorial to demonstrate that our approach to catalytic
reactions leads to enhanced diffusion of the catalyzer.  When you
begin, the code is incomplete and will produce errors when evaluated
in \es.  It needs your input to function properly.  A fully functional
file exists in the `SOLUTIONS' folder, but we recommend that you try
solving the exercises first.

To start the exercises, go into the `EXERCISES' directory and invoke
\es\ on the script
\begin{bash}
$ ./../build/Espresso reaction.tcl 0
\end{bash}
where the parameter \code{0} determines whether the reaction is
enabled.  Here, 0/1 corresponds to reaction off/on, just like on your
VCR.  At this stage, executing the above line will cause an error, as
the exercise has not yet been completed.

\subsection{Structure of the Simulation}

Let's walk through the script.  It is best to open the file while
reading this.

First, we read the activity parameter from the command line and verify
it.  Then we set up some general simulation parameters, such as box
length, radius of the colloid, concentration of reactant and products,
and the reaction rate.  Next, we setup the thermostat parameters, but
do not enable it yet.

Before we can set up the colloid and the small particles around, the
first two exercises have to be completed.

The script continues to setup interactions between the colloid and the
smaller particles and, most importantly, the reaction.  The syntax for
setting up a reaction is given above in the section ``Usage in \es''.

Warmup is performed by the \codees{minimize_energy} routine.  It has
several advantages over traditional force-capping warmup, which you
will learn about when completing the associated exercise.

Now the thermostat is enabled and the equilibration is performed.
Finally, we perform five production runs in which we measure the
mean-square displacement (MSD) and the angular velocity
auto-correlation function (AVACF).

You should have learned about everything that does not have to with
catalytic reactions in previous tutorials.  If you are unfamiliar with
any concept, it might be better to go back and complete the other
tutorials first.

\subsection{What to Expect}

Once you have completed all the tasks in the exercise script, the time
has come to run the actual simulation.  Run the script twice, once
with the activity parameter set to 0 and once set to 1.  You will have
two directories: \code{active-system} and \code{passive-system}.
These contain MSD and AVACF data files numbered consecutively over the
different runs.

It is now your job to average over these five files.  You can do this
in \code{gnuplot} or the scripting language of your choice.  The MSD
files have five columns each.  The first two columns are time and
samples, the last three are $x$, $y$, $z$.  The AVACF files have only
three columns, where the first two are also time and samples and the
third is the AVACF averaged over all spatial components.  For the MSD
we still have to average over the three spatial dimensions.

When we have extracted the mean (and recommendably the standard error)
we can plot it over time to achieve plots like in
figures~\ref{fig:msd} and~\ref{fig:avacf}.  We can clearly see, that
the reaction facilitates enhanced translational diffusion while
leaving the rotational diffusion unaffected.

\begin{figure}[tb]
  \centering
  \leavevmode\hfill
  \begin{minipage}[t]{.45\linewidth}
    \centering
    \includegraphics{FIGURES/msd}
    \caption{Averaged MSD over five runs with standard error on the
      error bars for both, the active and the passive system.  The
      black lines serve as a guide to the eye and indicate the
      dependence of the MSD on the time $t$.}
    \label{fig:msd}
  \end{minipage}
  \hfill
  \begin{minipage}[t]{.45\linewidth}
    \centering
    \includegraphics{FIGURES/avacf}
    \caption{The AVACF for the same system as in
      figure~\ref{fig:msd}. Note that the activity does not influence
      the rotational behavior.}
    \label{fig:avacf}
  \end{minipage}
  \hfill\null
\end{figure}


\section{Concluding Remarks}

With that, you have come to the end of this tutorial. We hope you
found it informative and that you have a sufficient understanding of
the way to deal with catalytic reactions in \es{} to set up
simulations on your own.

\section*{References}

\bibliographystyle{unsrt}
\bibliography{refs}

\end{document}

%%% Local Variables: 
%%% mode: latex
%%% End:
